\section{Morphological closing of tubular structures}

\subsection{Introduction}
As the overall aim of the project is to trace the blood vessel network, we are not interested in the tubes' hollow structure. In fact, for segmentation it would be easier if the tubes were plain, because then we would not have to deal with dark ''none-tube-voxels'' inside bright ''tube-voxels''. Our first task is thus to try to ''fill'' the tubes, using grayscale morphological closing (\cite{vincent1993morphological}.

\subsection{Workflow}

\begin{description}
\item[Data examination] \hfill \\ 
Open and view the data in Fiji using the following commands:

\item Open file: \ijmenu{File > Open...}  ''../bloodvessels\_small.tif''
\item View in 3D: \ijmenu{Plugins > 3D Viewer}
\item Change brightness (in 3D viewer menu): \ijmenu{[Edit > Transfer Function > RGBA]}
\item[Perform 3D morphological closing] \hfill \\ 
As the resolution of the dataset can be assumed reasonably isotropic and since the tubes can have any orientation we will use a spherical structuring element for the morphological closing. Select the CloseGray filter in \ijmenu{Plugins > 3D > 3D Fast Filters} with same kernel radius in all dimensions. A closing radius of 6 to 8 micrometers (3 to 4 voxels) is a sensible value for the dataset. This will not completely close the largest vessels however increasing the closing radius might merge the closest small vessels, so you have to go for a compromise here.  If you have time it is very instructive to also perform this grayscale closing operation by manually performing first a Maximum filter followed by Minimum filter.
\end{description}

\subsection{Generate an ImageJ macro}

Implement a macro performing above operations.

\underline{\textbf{Note}}: In the dialog box of the filters it is possible to input the radius in physical units or in pixels but only the radius in pixel shows up in the macro recorder. As it is convenient to input a radius in physical units you could write code to convert from micrometers to pixel units before calling the filter. For this you will require the macro function \ijmacro{getPixelSize}. In general, to combine numbers with the text strings as ImageJ plugin arguments you need \ijmacro{d2s(m,n)}, which converts a number m to a string keeping n decimals.

\lstinputlisting{code/TubeAnalyst-1.ijm}