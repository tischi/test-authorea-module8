\section{Prefiltering to enhance filamentous voxels}

\subsection{Introduction}
The datasets used here exhibit a high contrast so that a simple intensity based thresholding is almost sufficient to distinguish tube from background voxels. However in case of higher noise and/or uneven sample staining one may need to filter the data prior to thresholding. A good criterion to follow for this operation is to notice that a voxel is part of a filament if there is one direction along which the intensity is quite constant (along the filament) and two perpendicular directions along which the intensities quickly drop (perpendicular to the filament). The ImageJ command \ijmenu{Plugins > Analyze > Tubeness} computes a metric reflecting to what extent a voxel and its local neighborhood fulfill this criterion. The implemented algorithm is based on \cite{Sato1998}.

\subsection{Workflow}
Select the output image of above section (''Closed.tif'').

\begin{description}

\item[Enhance filamentous voxels]\hfill\\
Use the \ijmenu{Plugins > Analyze > Tubeness} command on the data (after the morphological closing) and check the result for different "Sigma", which controls the size of a Gaussian filter that is applied before the actual ''Tubeness'' computation.
This Gaussian pre-filtering indirectly determines the size of the neighbourhood taken into account for computation of the local intensity distribution.

Sensible values are in the range of 6 to 8 micrometers but you can experiment with different values. It is in fact usually almost impossible to find a value that is optimal for both the smallest and the largest vessels. 

You will notice that the contrast is greatly enhanced after the filtering but voxels close to vessel branch points might be forced to zero as their neighborhood do not strictly follow the definition of being filamentous. If this problem is too pronounced it is possible to perform another pass of morphological closing after the pre-filtering to "repair" these gaps in the network.
\end{description}

\subsection{Generate an ImageJ macro script}
Implement a macro performing above operations.

\lstinputlisting{code/TubeAnalyst-2.ijm}

\section{Segmentation of tubular structures}

\subsection{Introduction}
In order to analyze the tubular network we need to decide whether a voxel is a part of a tube or part of the background. To do so we will threshold the data, i.e. assign voxels below or above a certain gray value as background or object. Typically such thresholding also yields spurious isolated voxels that are not part of the tubular network. We will clean up such voxels based on the criterion that they are rather isolated and not connected to many other voxels.

\subsection{Workflow}

\begin{description}
\item[Convert previous ("Tubeness") image to 8-bit]\hfill\\
This step is optional but it somewhat simplifies the following thresholding operation. You should be careful not to clip the intensities during the conversion: The easiest way is to find the voxel with minimum and maximum intensities in the stack by inspecting the stack histogram and setting the minimum and maximum intensity values of the display accordingly before the conversion with \ijmenu{Image > Adjust > Brightness/Contrast...} . 
\item[Generate a binary image]\hfill\\
Threshold the previous ("Tubeness") image using \ijmenu{Image > Adjust > Threshold} (manual, global thresholding). We recommend to convert the image to 8 bit before thresholding as the Adjust Threshold interface works better with 8 bit than with 32-bit format. The aim is to adjust the lower bound of the threshold so that most of the vessels are thresholded without getting merged (if you followed the previous 8-bit conversion step a lower bound intensity around 8 gray values should work fine for both datasets).

\textbf{\underline{Optional}}: Automated thresholding methods

If you have time you may explore some automated thresholding methods such as:
\begin{itemize}
\item \ijmenu{Image  Adjust > Auto Threshold}
\item \ijmenu{Image > Adjust > Auto Local Threshold}
\end{itemize}

\item[Clean up small objects] \hfill\\
Clean up object voxels that are isolated, i.e. not connected to a minimum number of neighboring object voxels. This can be done by using the "minimum volume" option of \ijmenu{Analyze > 3D Objects Counter}. You can compare the initial segmentation mask and the resulting label mask after running "3D Objects Counter" in order to see what objects have been discarded. A practical value for the minimum volume filter is around 1000 voxels but you can experiment with different values.

\textbf{\underline{Note}}: The voxels of the output have an intensity corresponding to the index of the connected object they are part of (label mask). The indexing starts at 1 and, depending on the active Look-Up Table (LUT), some objects can thus appear very faint. To better visualize the results it is handy to use the "Random.lut" (see section \ref{sec:mod8prereq}). Just select it from \ijmenu{Image > Lookup Tables} or call \ijmacro{run("Random")} from your macro script.

\end{description}

\subsection{Generate an ImageJ macro script}
Write a macro performing above operations. The lower bound of the threshold should be stored in a variable \textbf{VesselThreshold} and the minimum volume for each connected components should be called \textbf{VesselVolumeThreshold}.

\textbf{\underline{Note}} : For a proper 8-bit conversion of the Tubeness image you need to set the display to the minimum and maximum gray value of the image stack. In a macro the minimum and maximum value of the stack can be retrieved using \ijmacro{Stack.getStatistics(voxelCount, min, max, mean, std)} and the current bounds of the display can be set by \ijmacro{setMinAndMax(min, max)}.

\lstinputlisting{code/TubeAnalyst-3.ijm}
